\documentclass{article}
\usepackage{amsmath, amsfonts, amssymb, amsthm, array, enumitem, needspace}
\usepackage[parfill]{parskip}
\usepackage[german]{babel}
\usepackage{color}   %May be necessary if you want to color links
\usepackage{hyperref}
\hypersetup{
	colorlinks=true, %set true if you want colored links
	linktoc=all,     %set to all if you want both sections and subsections linked
	linkcolor=black,  %choose some color if you want links to stand out
}

\newcommand\func[5]{%
	\begingroup
	\setlength\arraycolsep{0pt}
	#1\colon\begin{array}[t]{l >{{}}c<{{}} l}
		#2 & \to & #3 \\ #4 & \mapsto & #5 
	\end{array}%
	\endgroup}
	
\newtheorem{thm}{Satz}[section]
\newtheorem{lem}[thm]{Lemma}
\newtheorem{cor}[thm]{Schlussfolgerung}
\newtheorem{rem}[thm]{Bemerkung}
\newtheorem{remark}[thm]{Bemerkung}
\newtheorem{conj}[thm]{Annahme}
\newtheorem{defn}{Definition}[section]
\renewcommand\qedsymbol{QED}
\counterwithin*{equation}{thm}

\makeatletter
\newcommand*{\rom}[1]{\expandafter\@slowromancap\romannumeral #1@}
\makeatother

\newenvironment{aleq}{
\begin{equation}
\begin{aligned}
}{
\end{aligned}
\end{equation}
}

\newenvironment{aleq*}{\begin{equation*}\begin{aligned}}{\end{aligned}\end{equation*}}

\title{Mathe-Ergänzungskurs}
\author{Linus Yury Schneeberg}
\date{2025-2027}
\begin{document}
	\maketitle
	\tableofcontents
	\newpage
	
	\part{Q1}
	\section{Reelle Zahlenfolgen}
	\subsection{Definitionen}
	\begin{defn}[Reelle Zahlenfolge]
		\quad\newline
	\(\func{a}{\mathbb{N}}{\mathbb{R}}{n}{a(n)=a_n}\) heißt reelle Zahlenfolge.
	\end{defn}
	\begin{defn}[Bildungsvorschrift]
		Als Bildungsvorschrift bezeichnet man 
		\begin{enumerate}[label=(\alph*)]
			\item \(a(n) = f(n)\) \ z.B. \(a(n) = n^2\) \ (explizit)
			\item \(a(n) = f(a_1, \dots, a_{n-1}, n)\) \ z.B. \(a(n+1) = a(n) + a(n-1)\) \ (rekursiv)
		\end{enumerate}
	\end{defn}
	
	\begin{defn}[Monotonie]
		\label{defMonotonie}
	Eine beliebige Folge \((a_n)\) ist...
	\begin{enumerate}
		\item ...monton steigend genau dann, wenn
		\begin{aleq*}
			\forall n_1,n_2 \in \mathbb{N} \colon n_1 > n_2 \implies a_{n_1} \geq a_{n_2} \text{.}
		\end{aleq*}
		
		\item ...monoton fallend genau dann, wenn
		\begin{aleq*}
			\forall n_1,n_2 \in \mathbb{N} \colon n_1 > n_2 \implies a_{n_1} \leq a_{n_2} \text{.}
		\end{aleq*}
		
		\item ...streng monoton steigend genau dann, wenn
		\begin{aleq*}
			\forall n_1,n_2 \in \mathbb{N} \colon n_1 > n_2 \implies a_{n_1} > a_{n_2} \text{.}
		\end{aleq*}
		
		\item ...streng monoton fallend genau dann, wenn
		\begin{aleq*}
			\forall n_1,n_2 \in \mathbb{N} \colon n_1 > n_2 \implies a_{n_1} < a_{n_2} \text{.}
		\end{aleq*}
	\end{enumerate}
	\end{defn}
	
	\begin{defn}[Beschränktheit]
		\label{defBeschraenktheit}
		Eine beliebige Folge \((a_n)\) ist...
		\begin{enumerate}
			\item ...nach unten beschränkt genau dann, wenn
			\begin{aleq*}
				\exists a \in \mathbb{R} \colon \forall n \in \mathbb{N} \colon a_n \geq a \text{.}
			\end{aleq*}
			\item ...nach oben beschränkt genau dann, wenn
			\begin{aleq*}
				\exists b \in \mathbb{R} \colon \forall n \in \mathbb{N} \colon a_n \leq b \text{.}
			\end{aleq*}
			\item ...beschränkt genau dann, wenn sie nach oben und nach unten beschränkt ist.
		\end{enumerate}
	\end{defn}
	
	\begin{defn}[Supremum]
		\label{defSupremum}
		Das Supremum einer beliebigen nach oben beschränkten Folge \((a_n)\) ist die kleinste obere Schranke dieser Folge.
	\end{defn}
	
	\begin{defn}[Infimum]
		\label{defInfimum}
		Analog zum Supremum ist das Infimum einer beliebigen nach unten beschränkten Folge \((a_n)\) die größte untere Schranke dieser Folge.
	\end{defn}
	
	\begin{defn}[Konvergenz]
		\label{defKonvergenz}
		Eine beliebige Folge \((a_n)\) ist konvergent gegen \(g\) genau dann, wenn
		\begin{aleq*}
			\forall \varepsilon > 0 \colon \exists N_\varepsilon \in \mathbb{N} \colon \forall n \geq N_\varepsilon \colon |a_n - g| < \varepsilon \text{.}
		\end{aleq*}
	\end{defn}
	
	\subsection{Satz (Der Grenzwert von \(a_n = \frac{1}{n}\) ist 0)}
	\begin{thm}
		\label{lim1/n}
		Sei \((a_n)_{n=1}^\infty\) eine Folge mit der Bildungsvorschrift \(a_n = \frac{1}{n}\). \\ Dann gilt	\(\lim\limits_{n \to \infty} a_n = 0\)
		
		\begin{proof}
			Die Behauptung ist per Definition der Konvergenz (Definition \ref{defKonvergenz}) äquivalent zu
			\begin{aleq*}
				&\forall \varepsilon > 0 \colon \exists N_\varepsilon \in \mathbb{N} \colon \forall n \geq N_\varepsilon \colon |a_n - 0| < \varepsilon \\
				\iff &\forall \varepsilon > 0 \colon \exists N_\varepsilon \in \mathbb{N} \colon \forall n \geq N_\varepsilon \colon \left|\frac{1}{n}\right| < \varepsilon
			\end{aleq*}
			
			Diese Aussage gilt, weil es für jedes \(\varepsilon\) ein \(N_\varepsilon\) gibt, so dass für alle \(n>N_\varepsilon\) der Betrag von \(\frac{1}{n}\) kleiner als \(\varepsilon\) ist. Dieses \(N_\varepsilon\) lässt sich durch \(\left\lceil \frac{1}{\varepsilon}\right\rceil + 1\) berechnen.
		\end{proof}
	\end{thm}
	
	\subsection{Satz (rekursive Summenfolge = explizite)}
	\begin{thm}
	Seien \(a_1(n)\) und \(a_2(n)\) Folgen mit den Bildungsforschriften
	\begin{aleq*}
			a_1(n) &= a_1(n) + (n + 1) & a_2(n) = \sum_{k=0}^{n} k \\
			a_1(0) &= 0 \text{.}
	\end{aleq*}
	\par
	Dann gilt \(\forall n \colon a_1(n) = a_2(n)\).
	\end{thm}
	
	\begin{proof} Der Beweis wird durch vollständige Induktion geführt. \\
	\textbf{Induktionsanfang:} Für \(n=0\)
	\begin{aleq}
		\label{a_1InductBase}
			a_1(0) &= 0 
	\end{aleq}
	\begin{aleq}
		\label{a_2InductBase}
			a_2(0) &= \sum_{k=0}^{0} k = 0
	\end{aleq}
	\begin{aleq*}
		(\ref{a_1InductBase}) \land (\ref{a_2InductBase}) \implies a_1(0) = a_2(0)
	\end{aleq*}
	
	
	\textbf{Induktionsschritt:} Induktionshypothese: \(\exists n \colon a_1(n) = a_2(n)\) \\
	Zu zeigen ist, \(\text{Ind. Hypot.} \implies a_1(n+1) = a_2(n+1)\)
	
	\begin{equation*}
		\begin{aligned}
			a_1(n+1) &= a_1(n) + (n+1) \\
			&= a_2(n) + (n+1) && \text{Ind. Hypot.} \\
			&= \sum_{k=0}^{n}k + (n+1) \\
			&= \sum_{k=0}^{n+1}k \\
			&= a_2(n+1)
		\end{aligned}
	\end{equation*}
	\end{proof}
	
	\subsection{Satz (Jede konvergente Folge ist beschränkt)}
	\begin{thm}
		Sei \((a_n)_{n=1}^{\infty}\) eine konvergente Folge mit dem Grenzwert \(a\). Dann gilt
		\[
		\exists m,M \in \mathbb{R} \colon \forall n \in \mathbb{N} \colon m < a_n < M \text{.}
		\]
		
		\begin{proof}
			Da \(a_n\) gegen \(a\) konvergiert gilt
		\[
		\forall \varepsilon>0 \colon \exists N_{\varepsilon} \in \mathbb{N} \colon \forall n \geq N_{\varepsilon} \colon |a_n - a| < \varepsilon \text{.}
		\]
		Da \(|a_n - a| < \varepsilon\) in der oberen Aussage äquivalent zu \(-x < a_n < x\) ist, gilt auch
		\[
		\forall \varepsilon>0 \colon \exists N_{\varepsilon} \in \mathbb{N} \colon \forall n \geq N_{\varepsilon} \colon -\varepsilon < a_n - a < \varepsilon \text{.}
		\]
		\par
		Für ein bestimmtes \(\varepsilon > 0\) existiert also ein \(N_{\varepsilon}\), so dass 
		für alle \(n \geq N_{\varepsilon}\) \(a_n\) beschränkt ist. Da es nur endlich viele Folgenglieder für \(n < N_{\varepsilon}\) gibt, lässt sich eine obere Grenze als \(max (\left\lbrace a_n | n < N_{\varepsilon} \right\rbrace  \cup \lbrace \varepsilon + a \rbrace)\) und eine untere Grenze als \(min (\left\lbrace a_n | n < N_{\varepsilon} \rbrace \cup \lbrace -\varepsilon + a \rbrace\right\rbrace)\) berechnen.
		\end{proof}
	\end{thm}
	
	\subsection{Satz von Bolzano-Weierstraß (\rom{1} und \rom{2})}
	\begin{thm}[Satz von Bolzano-Weierstraß \rom{1}]
		Jede beschränkte Folge hat eine konvergente Teilfolge.
		
		\begin{proof} \((a_n)_{n=1}^{\infty}\) sei beschränkt durch \(m \leq a_n \leq M\) für alle \(n \in \mathbb{N}\). Man teile das Intervall \([n,M]\) in zwei Teile bei \(\frac{m+M}{2}\).
			\begin{enumerate}[label=\theenumi . Fall:]
				\item Auf \(\frac{m+M}{2}\) liegen unendlich viele Folgeglieder.
				\item In \([m, \frac{m+M}{2}[\) liegen unendlich viele Folgeglieder. \\Dann beginne mit \([m, \frac{m+M}{2}[\) von vorne.
				\item In \(]\frac{m+M}{2}, M]\) liegen undenlich viele Folgeglieder. \\Dann beginne mit \(]\frac{m+M}{2}, M]\) von vorne.
			\end{enumerate}
			\newpage
			\par
			Das Verfahren...
			\begin{enumerate}[label=(\alph*)]
				\item ... bricht mit Eintreten des ersten Falls ab und hat damit eine konvergente Teilfolge.
				\item ... setzt sich unendlich fort und erzeugt eine Folge von Intervallen mit
				\begin{itemize}
					\item \(I_n \subset I_{n-1}, I_0 = [m;M]\),
					\item Länge von \(I_n = \frac{M-m}{2^n} \stackrel{n \to \infty}{\to} 0\),
					\item Jedes Intervall enthält unendlich viele Folgeglieder.
				\end{itemize}
			\end{enumerate}
			\par
			Zu dieser Intervallschachtelung gehört genau eine reelle Zahl. Nimmt man aus jedem Intervall das Folgeglied mit dem kleinsten Index, welches noch nicht vorher ausgewählt wurde, erhält man eine Teilfolge, die gegen diese Zahl konvergiert.
		\end{proof}
	\end{thm}
	
	\begin{thm}[Satz von Bolzano-Weierstraß \rom{2}]
		Jede beschränkte und monotone Folge ist konvergent.
		\begin{proof}
			O.B.d.A (Ohne Beschränkung der Allgemeinheit) sei \((a_n)_{n=1}^{\infty}\) monoton wachsend. Sei \(sup\) das Supremum von \((a_n)\).
			
			Weil \(sup\) das Supremum von \((a_n)\) ist, gilt
			\begin{aleq*}
				\forall n \in \mathbb{N} \colon a_n \leq sup \land \forall \varepsilon > 0 \colon \exists N_\varepsilon \in \mathbb{N} \colon sup - \varepsilon < a_{N_\varepsilon} \text{.}
			\end{aleq*}
			\par
			Wir zeigen nun, dass \((a_n)\) gegen \(sup\) konvergiert, mit anderen Worten:
			\begin{aleq}
				\label{konvergenz}
				\forall \varepsilon > 0 \colon \exists N_\varepsilon \colon \forall n > N_\varepsilon \colon |a_n - sup| < \varepsilon \text{.}
			\end{aleq}
			\par
			Es gilt
			\begin{aleq*}
				&|a_n - sup| < \varepsilon \\
				\iff &sup - a_n < \varepsilon && \text{weil } sup > a_n \\
				\iff &sup - \varepsilon < a_n
			\end{aleq*}
			\par
			Aussage (\ref{konvergenz}) ist also wahr genau dann, wenn
			\begin{aleq*}
				\forall \varepsilon > 0 \colon \exists N_\varepsilon \colon \forall n > N_\varepsilon \colon sup - \varepsilon < a_n \text{.}
			\end{aleq*}
			\par
			Laut Definition des Supremums (\ref{defSupremum}) gilt \(\forall \varepsilon > 0 \colon \exists N_\varepsilon \in \mathbb{N} \colon  sup - \varepsilon < a_{N_\varepsilon}\). Weil \((a_n)\) monoton wachsend ist, gilt auch \(\forall n > N_\varepsilon \colon a_{N_\varepsilon} \leq a_n\). Daraus folgt, dass (\ref{konvergenz}) wahr ist und \((a_n)\) gegen \(sup\) konvergiert.
		\end{proof}
	\end{thm}
	\subsection{Cauchy-Folgen}
	\begin{defn}
		Eine Folge \((a_n)_{n=1}^\infty\) heißt Cauchy-Folge (altmodisch auch Fundamentalfolge), wenn gilt:
		\begin{aleq*}
			\forall \varepsilon > 0 \colon \exists N_\varepsilon \in \mathbb{N} \colon \forall m,n \in \mathbb{N} \colon m,n \geq N_\varepsilon \implies |a_m -a_n| < \varepsilon
		\end{aleq*}
	\end{defn}
	
	\begin{lem}
		\label{betragSumme}
		\begin{aleq*}
			\forall x,y \in \mathbb{R} \colon |x+y| \leq |x| + |y|
		\end{aleq*}
		\begin{proof}
			Seien \(x,y \in \mathbb{R}\) und beliebig, aber fest.
			Da ein beliebiges \(a \in \mathbb{R}\) entweder positiv (\(a = |a|\)) oder negativ (\(a = -|a|\)) ist (und \(-a\) auch) gilt:
			\begin{aleq}
				a \leq |a| \land -a \leq |a|
			\end{aleq}
			\par
			Für \(x+y\) müssen zwei Fälle überprüft werden:
			\begin{enumerate}[label=\theenumi . Fall: ]
				\item \(x+y \geq 0\) \\
				\begin{aleq*}
					|x+y| = x+y \\
					(1) \implies x+y \leq |x| + |y|
				\end{aleq*}
				
				\item \(x_y < 0\) \\
				\begin{aleq*}
					|x+y| = -x-y \\
					(1) \implies -x-y \leq |x| + |y|
				\end{aleq*}
			\end{enumerate}
		\end{proof}
	\end{lem}
	
	\begin{thm}
		In den reellen Zahlen (in jeder topologisch abgeschlossenen Menge mit Abstandsbegriff) sind Konvergenz und Cauchy-Eigenschaft äquivalent.
		
		\begin{proof}
			(\(\implies\)) Sei \((a_n)_{n=1}^\infty\) konvergent gegen \(a\). Zu zeigen ist, dass \((a_n)\) eine Cauchy-Folge ist:
			\begin{aleq*}
				\forall \varepsilon > 0 \colon \exists N_\varepsilon \in \mathbb{N} \colon \forall m,n \in \mathbb{N} \colon m,n \geq N_\varepsilon \implies |a_m -a_n| < \varepsilon
			\end{aleq*}
			\par
			Wir wissen, dass \((a_n)\) gegen \(a\) konvergiert. Es gilt also
			\begin{aleq*}
				\forall \varepsilon > 0 \colon \exists N_\varepsilon \colon \forall n > N_\varepsilon \colon |a_n - a| < \varepsilon \text{.}
			\end{aleq*}
			\par
			Weil die Aussage für alle \(\varepsilon\) (also auch für \(\frac{\varepsilon}{2}\)) gilt, finden wir auch ein \(N_\varepsilon\) und ein \(M_\varepsilon\), so dass gilt
			\begin{aleq*}
				\forall \varepsilon > 0 \colon \forall n > N_\varepsilon, m > M_\varepsilon \colon |a_n - a| + | a_m - a| < \varepsilon \text{.}
			\end{aleq*}
			
			\par
			Laut Lemma \ref{betragSumme} gilt also auch
			\begin{aleq*}
				\forall \varepsilon > 0 \colon \forall n > N_\varepsilon, m > M_\varepsilon \colon |a_m - a + a - a_n| \leq |a_n - a| + | a_m - a| < \varepsilon \text{.}
			\end{aleq*}
			\par
			Das impliziert
			\begin{aleq*}
				\forall \varepsilon > 0 \colon \forall n > N_\varepsilon, m > M_\varepsilon \colon |a_m - a_n| < \varepsilon \text{.}
			\end{aleq*}
			\par
			Das ist äquivalent zur Definition der Cauchy-Folge, weil man ein \(K_\varepsilon\) bestimmen kann, welches größer oder gleich \(N_\varepsilon\) und \(M_\varepsilon\) ist. Man wähle also \(K_\varepsilon := max \left\lbrace N_\varepsilon, M_\varepsilon\right\rbrace\). Dann gilt
			\begin{aleq*}
				\forall \varepsilon > 0 \colon \forall n, m > K_\varepsilon \colon |a_m - a_n| < \varepsilon \text{.}
			\end{aleq*}
			\par
			Damit ist \((a_n)\) eine Cauchy-Folge.
			\par
			(\(\impliedby\)) Sei \((b_n)\) eine Cauchy-Folge. Dann gilt
			\begin{aleq*}
				&\forall \varepsilon > 0 \colon \exists N_\varepsilon \in \mathbb{N} \colon \forall m,n \geq N_\varepsilon \colon |b_m-b_n| < \varepsilon \text{.}
			\end{aleq*}
			
			Weil diese Aussage für \underline{alle} \(m,n \geq N_\varepsilon\) gilt, gilt sie auch für\\ \(n=N_\varepsilon,m\geq N_\varepsilon\). Das bedeutet, dass alle \(b_m\) nicht weiter von \(b_{N_\varepsilon}\) entfernt sind als \(\varepsilon\). Oder auch
			\begin{aleq*}
				\forall \varepsilon > 0 \colon \exists N_\varepsilon \in \mathbb{N} \colon \forall m \geq N_\varepsilon \colon |b_m - b_{N_\varepsilon}| < \varepsilon \text{.}
			\end{aleq*}
			
			Diese Aussage sei nicht zu verwechseln mit der Definition der Konvergenz. Der entscheidende Unterschied ist, dass \(a_{N_\varepsilon}\) kein fester Wert ist. Was allerdings aus dieser Aussage folgt ist, dass \((b_m)\) für \(m \geq N_\varepsilon\) beschränkt ist. Die Folge ist auch für alle \(m < N_\varepsilon\) beschränkt, weil es nur endlich viele Folgeglieder mit diesem Kriterium gibt. Es lässt sich also eine obere Schranke als \(max \left\lbrace b_n | n < N_\varepsilon \right\rbrace\) und eine untere Schranke als \(min \left\lbrace b_n | n < N_\varepsilon \right\rbrace\) berechnen.
			\dots
		\end{proof}
	\end{thm}
	
	\vdots
	\subsection{Einschachtelungssatz/Sandwichlemma}
	\begin{thm}[Einschachtelungssatz/Sandwichlemma]
		Seien \((a_n)_{n=1}^\infty\), \((b_n)_{n=1}^\infty\) und \((c_n)_{n=1}^\infty\) beliebige Folgen mit \(\forall n \in \mathbb{N} \colon a_n \leq b_n \leq c_n\) und \(\lim\limits_{n \to \infty} a_n = \lim\limits_{n \to \infty} b_n = g\). Dann konvergiert auch \((b_n)\) gegen \(g\).
		
		\begin{proof}
			Zu zeigen ist
			\begin{aleq*}
				\forall \varepsilon > 0 \colon \exists N_\varepsilon \in \mathbb{N} \colon \forall n > N_\varepsilon \colon |b_n - g| < \varepsilon \text{.}
			\end{aleq*}
			\par
			Gegeben ist
			\begin{aleq*}
				\label{gegebenSandwich}
				\forall \varepsilon > 0 \colon \exists N_\varepsilon, M_\varepsilon \in \mathbb{N} \colon \forall n > N_\varepsilon, m > M_\varepsilon \colon |a_n-g| < \varepsilon \land |c_m - g| < \varepsilon \text{.}
			\end{aleq*}
			\par
			Es existiert also für alle \(\varepsilon > 0\) ein \(N_\varepsilon\) und ein \(M_\varepsilon\), so dass für \(K_\varepsilon = max \left\lbrace N_\varepsilon, M_\varepsilon \right\rbrace\) gilt
			\begin{aleq*}
				\forall k > K_\varepsilon \colon |a_k-g| < \varepsilon \land |c_k - g| < \varepsilon \text{.}
			\end{aleq*}
			\par
			Das und \(\forall n \in \mathbb{N} \colon a_n \leq b_n \leq c_n\) implizieren
			\begin{aleq*}
				&\forall k > K_\varepsilon \colon -\varepsilon < a_k-g \leq b_k-g \leq c_k-g < \varepsilon \\
				\implies &\forall k > K_\varepsilon \colon -\varepsilon < b_k-g < \varepsilon \\
				\iff &\forall k > K_\varepsilon \colon |b_k - g| < \varepsilon
			\end{aleq*}
			\par
			\((b_n)\) ist also ebenfalls konvergent gegen \(g\).
		\end{proof}
	\end{thm}
	
	\subsection{Teilfolgekriterium}
	\begin{thm}[Teilfolgekriterium]
		Eine Folge \((a_n)_{n=1}^\infty\) konvergiert genau dann gegen \(g\), wenn jede Teilfolge von \((a_n)\) ebenfalls gegen \(g\) konvergiert.
		\begin{proof}
			\((\impliedby)\) Wenn jede Teilfolge von \((a_n)\) gegen \(g\) konvergiert, konvergiert auch \((a_n)\) gegen \(g\), weil \((a_n)\) eine Teilfolge von \((a_n)\) ist. \\
			\par
			\((\implies)\) Indirekter Beweis. \\
				\textbf{Annahme: } Es gibt eine Teilfolge, die nicht gegen \(g\) konvergiert. Dann existiert  eine streng monoton steigende Folge von natürlichen Zahlen \((n_k)_{k=1}^\infty\), so dass die Folge \(b_k = a_{n_k}\) eine Teilfolge von \((a_n)\) ist, für die gilt
			\begin{aleq}
				\label{nichtKonvergenz}
				&\lnot \forall \varepsilon > 0 \colon \exists K_\varepsilon \in \mathbb{N} \colon \forall k > K_\varepsilon \colon |b_k -g| < \varepsilon \\
				\iff &\lnot \forall \varepsilon > 0 \colon \exists K_\varepsilon \in \mathbb{N} \colon \forall k > K_\varepsilon \colon |a_{n_k} -g| < \varepsilon
			\end{aleq}
			\par
			Weil \((a_n)\) gegen \(g\) konvergiert, gilt
			\begin{aleq*}
				&\forall \varepsilon > 0 \colon \exists N_\varepsilon \in \mathbb{N} \colon \forall m > N_\varepsilon \colon |a_m - g| < \varepsilon \text{.}
			\end{aleq*}
			\par
			Weil \(n_m \geq m > N_\varepsilon\) (strenge Monotonie von \((n_k)\)), gilt
			\begin{aleq*}
				&\forall \varepsilon > 0 \colon \exists N_\varepsilon \in \mathbb{N} \colon \forall m > N_\varepsilon \colon |a_{n_m} - g| < \varepsilon \text{,}
			\end{aleq*}
			\par
			was im Widerspruch zu (\ref{nichtKonvergenz}) steht.
 		\end{proof}
	\end{thm}
	
	\subsection{Grenzwertsätze für Folgen}
	\begin{thm}[Grenzwertsätze]
		\label{satzGW}
		Seien \((a_n)\) und \((b_n)\) konvergente Folgen mit
		\begin{aleq*}
			\lim_{n \to \infty} a_n = a \land \lim_{n \to \infty} b_n = b \text{.}
		\end{aleq*}
		
		Dann gelten folgende Aussagen.
		\begin{enumerate}
			\item \(\lim\limits_{n \to \infty} a_n + b_n = a + b\)
			\item \(\lim\limits_{n \to \infty} a_n - b_n = a - b\)
			\item \(\lim\limits_{n \to \infty} a_n * b_n = a * b\)
			\item \(\lim\limits_{n \to \infty} \frac{a_n}{b_n} = \frac{a}{b}, b \neq 0 \land \forall n \in \mathbb{N} \colon b_n \neq 0\)
		\end{enumerate}
		
		\begin{proof}
			\quad\newline
			\begin{enumerate}
				\item Weil \((a_n)\) und \((b_n)\) gegen \(a\) bzw. \(b\) konvergieren (\(|a_n -a|\) und \(|b_n - b|\) werden beliebig klein), gilt
				\begin{aleq*}
					&\forall \varepsilon > 0 \colon \exists N_\varepsilon \colon \forall n \geq N_\varepsilon \colon |a_n - a | + |b_n - b| < \varepsilon \\
					\implies &\forall \varepsilon > 0 \colon \exists N_\varepsilon \colon \forall n \geq N_\varepsilon \colon |a_n - a + b_n - b| < \varepsilon && \text{Lemma \ref{betragSumme}} \\
					\iff &\forall \varepsilon > 0 \colon \exists N_\varepsilon \colon \forall n \geq N_\varepsilon \colon |a_n + b_n - (a + b)| < \varepsilon
				\end{aleq*}
				
				Das bedeutet, dass der Grenzwert von \(a_n + b_n = a+b\) ist. Damit ist 1. bewiesen.
				
				\item 
				\dots
				
				\item 
			\end{enumerate}
		\end{proof}
	\end{thm}
	
	\subsection{Übungsaufgaben}
	\begin{enumerate}
		\item Finden Sie den Grenzwert der jeweiligen Folge.
		\begin{enumerate}[label=(\alph*)]
			\item \(a_n = \frac{\frac{1}{n} + \frac{1}{n^2} + 1}{\frac{1}{n^3} + \frac{7}{n} + 3}\)
			\item \(b_n = \frac{n^2 + n - 1}{n^3 + 1}\)
			\item \(c_n = \frac{\left(\frac{1}{3}\right)^n + \left(\frac{1}{4}\right)^n + 1^n}{\left(\frac{1}{5}\right)^n + 2}\)
			\item \(d_n = \frac{3^n + 5^n + 7^n}{2^n + 3^n + 7^n}\)
		\end{enumerate}

		\item \needspace{2\baselineskip} Begründen Sie folgende Sätze mithilfe einer Skizze oder durch einen Beweis.
		\begin{enumerate}[label=(\alph*)]
			\item Eine Folge kann nur einen Grenzwert haben. \textit{(Hinweis: Indirekter Beweis)}
			\item Sei \((a_n)\) eine beliebige Folge für die \(\forall n \in \mathbb{N} \colon a_n > 0\) gilt (\((a_n)\) ist nach unten beschränkt, wobei eine untere Schranke \(0\) ist). Falls die Folge \((b_n)\) mit der Bildungsvorschrift \(b_n = (a_n)^n\) konvergiert kann \(\lim\limits_{n \to \infty} a_n < 0\) nicht stimmen.
		\end{enumerate}
	\end{enumerate}
	
	\subsubsection{Lösungen}
	\begin{enumerate}
		\item \quad
		\begin{enumerate}[label=(\alph*)]
			\item Wir wissen, dass \(\frac{1}{n}\) mit \(n \in \mathbb{N}\) gegen 0 konvergiert (Satz \ref{lim1/n}).
			\begin{aleq*}
				\lim_{n \to \infty} a_n = &\lim_{n \to \infty} \frac{\frac{1}{n} + \frac{1}{n^2} + 1}{\frac{1}{n^3} + \frac{7}{n} + 3} && \text{Satz \ref{lim1/n}} \\
				= &\lim_{n \to \infty} \frac{1}{3} && \text{Grenzwertsätze (\ref{satzGW})} \\
				= &\frac{1}{3}
			\end{aleq*}
		\end{enumerate}
	\end{enumerate}
	
\end{document}