\documentclass{article}
\usepackage{amsmath, amsfonts, amssymb, amsthm, array, enumitem}
\usepackage[parfill]{parskip}
\usepackage[german]{babel}

\newcommand\func[5]{%
	\begingroup
	\setlength\arraycolsep{0pt}
	#1\colon\begin{array}[t]{l >{{}}c<{{}} l}
		#2 & \to & #3 \\ #4 & \mapsto & #5 
	\end{array}%
	\endgroup}
	
\newtheorem{thm}{Satz}[section]
\newtheorem{lem}[thm]{Lemma}
\newtheorem{cor}[thm]{Schlussfolgerung}
\newtheorem{rem}[thm]{Bemerkung}
\newtheorem{remark}[thm]{Bemerkung}
\newtheorem{conj}[thm]{Annahme}
\newtheorem{defn}{Definition}[section]
\renewcommand\qedsymbol{QED}

\makeatletter
\newcommand*{\rom}[1]{\expandafter\@slowromancap\romannumeral #1@}
\makeatother

\newenvironment{aleq}{
\begin{equation}
\begin{aligned}
}{
\end{aligned}
\end{equation}
}

\newenvironment{aleq*}{\begin{equation*}\begin{aligned}}{\end{aligned}\end{equation*}}

\title{Mathe-Ergänzungskurs}
\author{Linus Yury Schneeberg}
\date{2025/26}
\begin{document}
	\maketitle
	\tableofcontents
	\newpage
	
	\section{Reelle Zahlenfolgen}
	\subsection{Definitionen}
	\begin{defn}
	$\func{a}{\mathbb{N}}{\mathbb{R}}{n}{a(n)=a_n}$ heißt reelle Zahlenfolge.
	\end{defn}
	\begin{defn}
	Als Bildungsvorschrift bezeichnet man 
	\begin{enumerate}[label=(\alph*)]
		\item \(a(n) = f(n)\) \ z.B. \(a(n) = n^2\) \ (explizit)
		\item \(a(n) = f(a_1, \dots, a_{n-1}, n)\) \ z.B. \(a(n+1) = a(n) + a(n-1)\) \ (rekursiv)
	\end{enumerate}
\end{defn}
	
	\subsection{Beweis (rekursive Summenfolge = explizite)}
	\begin{thm}
	Seien \(a_1(n)\) und \(a_2(n)\) Folgen mit den Bildungsforschriften
	\begin{aleq*}
			a_1(n) &= a_1(n) + (n + 1) & a_2(n) = \sum_{k=0}^{n} k \\
			a_1(0) &= 0 \text{.}
	\end{aleq*}
	\par
	Dann gilt \(\forall n \colon a_1(n) = a_2(n)\).
	\end{thm}
	
	\begin{proof} Der Beweis wird durch vollständige Induktion geführt. \\
	\textbf{Induktionsanfang:} Für \(n=0\)
	\begin{aleq}
			a_1(0) &= 0 
	\end{aleq}
	\begin{aleq}
			a_2(0) &= \sum_{k=0}^{0} k = 0
	\end{aleq}
	\begin{aleq*}
		(1) \land (2) \implies a_1(0) = a_2(0)
	\end{aleq*}
	
	
	\textbf{Induktionsschritt:} Induktionshypothese: \(\exists n \colon a_1(n) = a_2(n)\) \\
	Zu zeigen ist, \(\text{Ind. Hypot.} \implies a_1(n+1) = a_2(n+1)\)
	
	\begin{equation*}
		\begin{aligned}
			a_1(n+1) &= a_1(n) + (n+1) \\
			&= a_2(n) + (n+1) && \text{Ind. Hypot.} \\
			&= \sum_{k=0}^{n}k + (n+1) \\
			&= \sum_{k=0}^{n+1}k \\
			&= a_2(n+1)
		\end{aligned}
	\end{equation*}
	\end{proof}
	
	\subsection{Satz (Jede konvergente Folge ist beschränkt)}
	\begin{thm}
		Sei \((a_n)_{n=1}^{\infty}\) eine konvergente Folge mit dem Grenzwert \(a\). Dann gilt
		\[
		\exists m,M \in \mathbb{R} \colon \forall n \in \mathbb{N} \colon m < a_n < M
		\]
		
		\begin{proof}
			Da \(a_n\) gegen \(a\) konvergiert gilt
		\[
		\forall \varepsilon>0 \colon \exists N_{\varepsilon} \in \mathbb{N} \colon \forall n \geq N_{\varepsilon} \colon |a_n - a| < \varepsilon \text{.}
		\]
		Da \(|a_n - a| < \varepsilon\) in der oberen Aussage äquivalent zu \(-x < a_n < x\) ist, gilt auch
		\[
		\forall \varepsilon>0 \colon \exists N_{\varepsilon} \in \mathbb{N} \colon \forall n \geq N_{\varepsilon} \colon -\varepsilon < a_n - a < \varepsilon \text{.}
		\]
		\par
		Für ein bestimmtes \(\varepsilon > 0\) existiert also ein \(N_{\varepsilon}\), so dass 
		für alle \(n \geq N_{\varepsilon}\) \(a_n\) beschränkt ist. Da es nur endlich viele Folgenglieder für \(n < N_{\varepsilon}\) gibt, lässt sich eine obere Grenze als \(max (\left\lbrace a_n | n < N_{\varepsilon} \right\rbrace  \cup \lbrace \varepsilon + a \rbrace)\) und eine untere Grenze als \(min (\left\lbrace a_n | n < N_{\varepsilon} \rbrace \cup \lbrace -\varepsilon + a \rbrace\right\rbrace)\) berechnen.
		\end{proof}
	\end{thm}
	
	\subsection{Satz von Bolzano-Weierstraß}
	\begin{thm}[Satz von Bolzano-Weierstraß \rom{1}]
		Jede beschränkte Folge hat eine konvergente Teilfolge.
		
		\begin{proof} \((a_n)_{n=1}^{\infty}\) sei beschränkt durch \(m \leq a_n \leq M\) für alle \(n \in \mathbb{N}\). Man teile das intervall \([n,M]\) in zwei Teile bei \(\frac{m+M}{2}\).
			\begin{enumerate}[label=\theenumi . Fall:]
				\item Auf \(\frac{m+M}{2}\) liegen unendlich viele Folgeglieder.
				\item In \([m, \frac{m+M}{2}[\) liegen unendlich viele Folgeglieder. Dann beginne mit \([m, \frac{m+M}{2}[\) von vorne.
				\item In \(]\frac{m+M}{2}, M]\) liegen undenlich viele Folgeglieder. Dann beginne mit \(]\frac{m+M}{2}, M]\) von vorne.
			\end{enumerate}
			
		\end{proof}
	\end{thm}
	
\end{document}