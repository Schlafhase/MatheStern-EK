\documentclass{article}
\usepackage{amsmath, amsfonts, amssymb, amsthm, array, enumitem, needspace}
\usepackage[parfill]{parskip}
\usepackage[german]{babel}
\usepackage{color}   %May be necessary if you want to color links
\usepackage[svgnames]{xcolor}
\usepackage{hyperref}
\hypersetup{
	colorlinks=true, %set true if you want colored links
	linktoc=all,     %set to all if you want both sections and subsections linked
	linkcolor=black,  %choose some color if you want links to stand out
}

\newcommand\func[5]{%
	\begingroup
	\setlength\arraycolsep{0pt}
	#1\colon\begin{array}[t]{l >{{}}c<{{}} l}
		#2 & \to & #3 \\ #4 & \mapsto & #5 
	\end{array}%
	\endgroup}
	
\newtheorem{thm}{Satz}[section]
\newtheorem{lem}[thm]{Lemma}
\newtheorem{cor}[thm]{Schlussfolgerung}
\newtheorem{rem}[thm]{Bemerkung}
\newtheorem{remark}[thm]{Bemerkung}
\newtheorem{conj}[thm]{Annahme}
\newtheorem{defn}{Definition}[section]
\renewcommand\qedsymbol{QED}
\counterwithin*{equation}{thm}

\makeatletter
\newcommand*{\rom}[1]{\expandafter\@slowromancap\romannumeral #1@}
\makeatother

\newenvironment{aleq}{
\begin{equation}
\begin{aligned}
}{
\end{aligned}
\end{equation}
}

\newenvironment{aleq*}{\begin{equation*}\begin{aligned}}{\end{aligned}\end{equation*}}

\newcommand{\TODO}{
	
	{
		\color{DarkRed}
		\fbox{\begin{minipage}{\textwidth}
				\textbf{Hinweis} \\
				An dieser Stelle fehlt (im Moment) etwas. Falls Sie dazu beitragen möchten dieses Dokument zu vervollständigen, können Sie dies auf \href{https://github.com/Schlafhase/MatheStern-EK/}{GitHub} tun.
		\end{minipage}}
	}

}

\newcommand{\LICENSE}{
	Copyright (C)  2025 Linus Yury Schneeberg.
	Permission is granted to copy, distribute and/or modify this document
	under the terms of the GNU Free Documentation License, Version 1.3
	or any later version published by the Free Software Foundation;
	with no Invariant Sections, no Front-Cover Texts, and no Back-Cover Texts.
	A copy of the license is included in the \hyperref[sec:fdl]{section entitled "GNU
	Free Documentation License"}.
}

\title{Mathe-Ergänzungskurs}
\author{Linus Yury Schneeberg}
\date{2025-2027}
\begin{document}
	\maketitle
	\tableofcontents
	\newpage
	\LICENSE
	\newpage
	
	\part{Q1}
	\section{Reelle Zahlenfolgen}
	\subsection{Definitionen}
	\begin{defn}[Reelle Zahlenfolge]
		\quad\newline
	\(\func{a}{\mathbb{N}}{\mathbb{R}}{n}{a(n)=a_n}\) heißt reelle Zahlenfolge.
	\end{defn}
	\begin{defn}[Bildungsvorschrift]
		Als Bildungsvorschrift bezeichnet man 
		\begin{enumerate}[label=(\alph*)]
			\item \(a(n) = f(n)\) \ z.B. \(a(n) = n^2\) \ (explizit)
			\item \(a(n) = f(a_1, \dots, a_{n-1}, n)\) \ z.B. \(a(n+1) = a(n) + a(n-1)\) \ (rekursiv)
		\end{enumerate}
	\end{defn}
	
	\begin{defn}[Monotonie]
		\label{defMonotonie}
	Eine beliebige Folge \((a_n)\) ist...
	\begin{enumerate}
		\item ...monton steigend genau dann, wenn
		\begin{aleq*}
			\forall n_1,n_2 \in \mathbb{N} \colon n_1 > n_2 \implies a_{n_1} \geq a_{n_2} \text{.}
		\end{aleq*}
		
		\item ...monoton fallend genau dann, wenn
		\begin{aleq*}
			\forall n_1,n_2 \in \mathbb{N} \colon n_1 > n_2 \implies a_{n_1} \leq a_{n_2} \text{.}
		\end{aleq*}
		
		\item ...streng monoton steigend genau dann, wenn
		\begin{aleq*}
			\forall n_1,n_2 \in \mathbb{N} \colon n_1 > n_2 \implies a_{n_1} > a_{n_2} \text{.}
		\end{aleq*}
		
		\item ...streng monoton fallend genau dann, wenn
		\begin{aleq*}
			\forall n_1,n_2 \in \mathbb{N} \colon n_1 > n_2 \implies a_{n_1} < a_{n_2} \text{.}
		\end{aleq*}
	\end{enumerate}
	\end{defn}
	
	\begin{defn}[Beschränktheit]
		\label{defBeschraenktheit}
		Eine beliebige Folge \((a_n)\) ist...
		\begin{enumerate}
			\item ...nach unten beschränkt genau dann, wenn
			\begin{aleq*}
				\exists a \in \mathbb{R} \colon \forall n \in \mathbb{N} \colon a_n \geq a \text{.}
			\end{aleq*}
			\item ...nach oben beschränkt genau dann, wenn
			\begin{aleq*}
				\exists b \in \mathbb{R} \colon \forall n \in \mathbb{N} \colon a_n \leq b \text{.}
			\end{aleq*}
			\item ...beschränkt genau dann, wenn sie nach oben und nach unten beschränkt ist.
		\end{enumerate}
	\end{defn}
	
	\begin{defn}[Supremum]
		\label{defSupremum}
		Das Supremum einer beliebigen nach oben beschränkten Folge \((a_n)\) ist die kleinste obere Schranke dieser Folge.
	\end{defn}
	
	\begin{defn}[Infimum]
		\label{defInfimum}
		Analog zum Supremum ist das Infimum einer beliebigen nach unten beschränkten Folge \((a_n)\) die größte untere Schranke dieser Folge.
	\end{defn}
	
	\begin{defn}[Konvergenz]
		\label{defKonvergenz}
		Eine beliebige Folge \((a_n)\) ist konvergent gegen \(g\) genau dann, wenn
		\begin{aleq*}
			\forall \varepsilon > 0 \colon \exists N_\varepsilon \in \mathbb{N} \colon \forall n \geq N_\varepsilon \colon |a_n - g| < \varepsilon \text{.}
		\end{aleq*}
	\end{defn}
	
	\subsection{Satz (Der Grenzwert von \(a_n = \frac{1}{n}\) ist 0)}
	\begin{thm}
		\label{lim1/n}
		Sei \((a_n)_{n=1}^\infty\) eine Folge mit der Bildungsvorschrift \(a_n = \frac{1}{n}\). \\ Dann gilt	\(\lim\limits_{n \to \infty} a_n = 0\)
		
		\begin{proof}
			Die Behauptung ist per Definition der Konvergenz (Definition \ref{defKonvergenz}) äquivalent zu
			\begin{aleq*}
				&\forall \varepsilon > 0 \colon \exists N_\varepsilon \in \mathbb{N} \colon \forall n \geq N_\varepsilon \colon |a_n - 0| < \varepsilon \\
				\iff &\forall \varepsilon > 0 \colon \exists N_\varepsilon \in \mathbb{N} \colon \forall n \geq N_\varepsilon \colon \left|\frac{1}{n}\right| < \varepsilon
			\end{aleq*}
			
			Diese Aussage gilt, weil es für jedes \(\varepsilon\) ein \(N_\varepsilon\) gibt, so dass für alle \(n>N_\varepsilon\) der Betrag von \(\frac{1}{n}\) kleiner als \(\varepsilon\) ist. Dieses \(N_\varepsilon\) lässt sich durch \(\left\lceil \frac{1}{\varepsilon}\right\rceil + 1\) berechnen.
		\end{proof}
	\end{thm}
	
	\subsection{Satz (rekursive Summenfolge = explizite)}
	\begin{thm}
	Seien \(a_1(n)\) und \(a_2(n)\) Folgen mit den Bildungsforschriften
	\begin{aleq*}
			a_1(n) &= a_1(n) + (n + 1) & a_2(n) = \sum_{k=0}^{n} k \\
			a_1(0) &= 0 \text{.}
	\end{aleq*}
	\par
	Dann gilt \(\forall n \colon a_1(n) = a_2(n)\).
	\end{thm}
	
	\begin{proof} Der Beweis wird durch vollständige Induktion geführt. \\
	\textbf{Induktionsanfang:} Für \(n=0\)
	\begin{aleq}
		\label{a_1InductBase}
			a_1(0) &= 0 
	\end{aleq}
	\begin{aleq}
		\label{a_2InductBase}
			a_2(0) &= \sum_{k=0}^{0} k = 0
	\end{aleq}
	\begin{aleq*}
		(\ref{a_1InductBase}) \land (\ref{a_2InductBase}) \implies a_1(0) = a_2(0)
	\end{aleq*}
	
	
	\textbf{Induktionsschritt:} Induktionshypothese: \(\exists n \colon a_1(n) = a_2(n)\) \\
	Zu zeigen ist, \(\text{Ind. Hypot.} \implies a_1(n+1) = a_2(n+1)\)
	
	\begin{equation*}
		\begin{aligned}
			a_1(n+1) &= a_1(n) + (n+1) \\
			&= a_2(n) + (n+1) && \text{Ind. Hypot.} \\
			&= \sum_{k=0}^{n}k + (n+1) \\
			&= \sum_{k=0}^{n+1}k \\
			&= a_2(n+1)
		\end{aligned}
	\end{equation*}
	\end{proof}
	
	\subsection{Satz (Jede konvergente Folge ist beschränkt)}
	\begin{thm}
		\label{konvergentBeschraenkt}
		Sei \((a_n)_{n=1}^{\infty}\) eine konvergente Folge mit dem Grenzwert \(a\). Dann gilt
		\[
		\exists m,M \in \mathbb{R} \colon \forall n \in \mathbb{N} \colon m < a_n < M \text{.}
		\]
		
		\begin{proof}
			Da \(a_n\) gegen \(a\) konvergiert gilt
		\[
		\forall \varepsilon>0 \colon \exists N_{\varepsilon} \in \mathbb{N} \colon \forall n \geq N_{\varepsilon} \colon |a_n - a| < \varepsilon \text{.}
		\]
		Da \(|a_n - a| < \varepsilon\) in der oberen Aussage äquivalent zu \(-x < a_n < x\) ist, gilt auch
		\[
		\forall \varepsilon>0 \colon \exists N_{\varepsilon} \in \mathbb{N} \colon \forall n \geq N_{\varepsilon} \colon -\varepsilon < a_n - a < \varepsilon \text{.}
		\]
		\par
		Für jedes \(\varepsilon > 0\) existiert also ein \(N_{\varepsilon}\), so dass \(a_n\) für alle \(n \geq N_{\varepsilon}\) beschränkt ist. Da es nur endlich viele Folgenglieder für \(n < N_{\varepsilon}\) gibt, lässt sich eine obere Grenze als 
		\begin{aleq*}
			max (\left\lbrace a_n | n < N_{\varepsilon} \right\rbrace  \cup \lbrace \varepsilon + a \rbrace)
		\end{aleq*}
		 und eine untere Grenze als 
		 \begin{aleq*}
		 	min (\left\lbrace a_n | n < N_{\varepsilon} \rbrace \cup \lbrace -\varepsilon + a \rbrace\right\rbrace)
		 \end{aleq*}
		  berechnen.
		\end{proof}
	\end{thm}
	
	\subsection{Satz von Bolzano-Weierstraß (\rom{1} und \rom{2})}
	\begin{thm}[Satz von Bolzano-Weierstraß \rom{1}]
		Jede beschränkte Folge hat eine konvergente Teilfolge.
		
		\begin{proof} \((a_n)_{n=1}^{\infty}\) sei beschränkt durch \(m \leq a_n \leq M\) für alle \(n \in \mathbb{N}\). Man teile das Intervall \([n,M]\) in zwei Teile bei \(\frac{m+M}{2}\).
			\begin{enumerate}[label=\theenumi . Fall:]
				\item Auf \(\frac{m+M}{2}\) liegen unendlich viele Folgeglieder.
				\item In \([m, \frac{m+M}{2}[\) liegen unendlich viele Folgeglieder. \\Dann beginne mit \([m, \frac{m+M}{2}[\) von vorne.
				\item In \(]\frac{m+M}{2}, M]\) liegen undenlich viele Folgeglieder. \\Dann beginne mit \(]\frac{m+M}{2}, M]\) von vorne.
			\end{enumerate}
			\par
			Das Verfahren...
			\begin{enumerate}[label=(\alph*)]
				\item ... bricht mit Eintreten des ersten Falls ab und hat damit eine konvergente Teilfolge.
				\item ... setzt sich unendlich fort und erzeugt eine Folge von Intervallen mit
				\begin{itemize}
					\item \(I_n \subset I_{n-1}, I_0 = [m;M]\),
					\item Länge von \(I_n = \frac{M-m}{2^n} \stackrel{n \to \infty}{\to} 0\),
					\item Jedes Intervall enthält unendlich viele Folgeglieder.
				\end{itemize}
			\end{enumerate}
			\par
			Zu dieser Intervallschachtelung gehört genau eine reelle Zahl. Nimmt man aus jedem Intervall das Folgeglied mit dem kleinsten Index, welches noch nicht vorher ausgewählt wurde, erhält man eine Teilfolge, die gegen diese Zahl konvergiert.
		\end{proof}
	\end{thm}
	
	\begin{thm}[Satz von Bolzano-Weierstraß \rom{2}]
		Jede beschränkte und monotone Folge ist konvergent.
		\begin{proof}
			O.B.d.A (Ohne Beschränkung der Allgemeinheit) sei \((a_n)_{n=1}^{\infty}\) monoton wachsend. Sei \(sup\) das Supremum von \((a_n)\).
			
			Weil \(sup\) das Supremum von \((a_n)\) ist, gilt
			\begin{aleq*}
				\forall n \in \mathbb{N} \colon a_n \leq sup \land \forall \varepsilon > 0 \colon \exists N_\varepsilon \in \mathbb{N} \colon sup - \varepsilon < a_{N_\varepsilon} \text{.}
			\end{aleq*}
			\par
			Wir zeigen nun, dass \((a_n)\) gegen \(sup\) konvergiert, mit anderen Worten:
			\begin{aleq}
				\label{konvergenz}
				\forall \varepsilon > 0 \colon \exists N_\varepsilon \colon \forall n > N_\varepsilon \colon |a_n - sup| < \varepsilon \text{.}
			\end{aleq}
			\par
			Es gilt
			\begin{aleq*}
				&|a_n - sup| < \varepsilon \\
				\iff &sup - a_n < \varepsilon && \text{weil } sup > a_n \\
				\iff &sup - \varepsilon < a_n
			\end{aleq*}
			\par
			Aussage (\ref{konvergenz}) ist also wahr genau dann, wenn
			\begin{aleq*}
				\forall \varepsilon > 0 \colon \exists N_\varepsilon \colon \forall n > N_\varepsilon \colon sup - \varepsilon < a_n \text{.}
			\end{aleq*}
			\par
			Laut Definition des Supremums (\ref{defSupremum}) gilt \(\forall \varepsilon > 0 \colon \exists N_\varepsilon \in \mathbb{N} \colon  sup - \varepsilon < a_{N_\varepsilon}\). Weil \((a_n)\) monoton wachsend ist, gilt auch \(\forall n > N_\varepsilon \colon a_{N_\varepsilon} \leq a_n\). Daraus folgt, dass (\ref{konvergenz}) wahr ist und \((a_n)\) gegen \(sup\) konvergiert.
		\end{proof}
	\end{thm}
	\subsection{Cauchy-Folgen}
	\begin{defn}
		Eine Folge \((a_n)_{n=1}^\infty\) heißt Cauchy-Folge (altmodisch auch Fundamentalfolge), wenn gilt:
		\begin{aleq*}
			\forall \varepsilon > 0 \colon \exists N_\varepsilon \in \mathbb{N} \colon \forall m,n \in \mathbb{N} \colon m,n \geq N_\varepsilon \implies |a_m -a_n| < \varepsilon
		\end{aleq*}
	\end{defn}
	
	\begin{lem}[Dreiecksungleichung]
		\label{dreiecksungleichung}
		\begin{aleq*}
			\forall x,y \in \mathbb{R} \colon |x+y| \leq |x| + |y|
		\end{aleq*}
		\begin{proof}
			Seien \(x,y \in \mathbb{R}\) und beliebig, aber fest.
			Da ein beliebiges \(a \in \mathbb{R}\) entweder positiv (\(a = |a|\)) oder negativ (\(a = -|a|\)) ist (und \(-a\) auch) gilt:
			\begin{aleq}
				a \leq |a| \land -a \leq |a|
			\end{aleq}
			\par
			Für \(x+y\) müssen zwei Fälle überprüft werden:
			\begin{enumerate}[label=\theenumi . Fall: ]
				\item \(x+y \geq 0\) \\
				\begin{aleq*}
					|x+y| = x+y \\
					(1) \implies x+y \leq |x| + |y|
				\end{aleq*}
				
				\item \(x_y < 0\) \\
				\begin{aleq*}
					|x+y| = -x-y \\
					(1) \implies -x-y \leq |x| + |y|
				\end{aleq*}
			\end{enumerate}
		\end{proof}
	\end{lem}
	
	\begin{thm}
		In den reellen Zahlen (in jeder topologisch abgeschlossenen Menge mit Abstandsbegriff) sind Konvergenz und Cauchy-Eigenschaft äquivalent.
		
		\begin{proof}
			(\(\implies\)) Sei \((a_n)_{n=1}^\infty\) konvergent gegen \(a\). Zu zeigen ist, dass \((a_n)\) eine Cauchy-Folge ist:
			\begin{aleq*}
				\forall \varepsilon > 0 \colon \exists N_\varepsilon \in \mathbb{N} \colon \forall m,n \in \mathbb{N} \colon m,n \geq N_\varepsilon \implies |a_m -a_n| < \varepsilon
			\end{aleq*}
			\par
			Wir wissen, dass \((a_n)\) gegen \(a\) konvergiert. Es gilt also
			\begin{aleq*}
				\forall \varepsilon > 0 \colon \exists N_\varepsilon \colon \forall n > N_\varepsilon \colon |a_n - a| < \varepsilon \text{.}
			\end{aleq*}
			\par
			Weil die Aussage für alle \(\varepsilon\) (also auch für \(\frac{\varepsilon}{2}\)) gilt, finden wir auch ein \(N_\varepsilon\) und ein \(M_\varepsilon\), so dass gilt
			\begin{aleq*}
				\forall \varepsilon > 0 \colon \forall n > N_\varepsilon, m > M_\varepsilon \colon |a_n - a| + | a_m - a| < \varepsilon \text{.}
			\end{aleq*}
			
			\par
			Laut Lemma \ref{dreiecksungleichung} gilt also auch
			\begin{aleq*}
				\forall \varepsilon > 0 \colon \forall n > N_\varepsilon, m > M_\varepsilon \colon |a_m - a + a - a_n| \leq |a_n - a| + | a_m - a| < \varepsilon \text{.}
			\end{aleq*}
			\par
			Das impliziert
			\begin{aleq*}
				\forall \varepsilon > 0 \colon \forall n > N_\varepsilon, m > M_\varepsilon \colon |a_m - a_n| < \varepsilon \text{.}
			\end{aleq*}
			\par
			Das ist äquivalent zur Definition der Cauchy-Folge, weil man ein \(K_\varepsilon\) bestimmen kann, welches größer oder gleich \(N_\varepsilon\) und \(M_\varepsilon\) ist. Man wähle also \(K_\varepsilon := max \left\lbrace N_\varepsilon, M_\varepsilon\right\rbrace\). Dann gilt
			\begin{aleq*}
				\forall \varepsilon > 0 \colon \forall n, m > K_\varepsilon \colon |a_m - a_n| < \varepsilon \text{.}
			\end{aleq*}
			\par
			Damit ist \((a_n)\) eine Cauchy-Folge.
			\par
			(\(\impliedby\)) Sei \((b_n)\) eine Cauchy-Folge. Dann gilt
			\begin{aleq*}
				&\forall \varepsilon > 0 \colon \exists N_\varepsilon \in \mathbb{N} \colon \forall m,n \geq N_\varepsilon \colon |b_m-b_n| < \varepsilon \text{.}
			\end{aleq*}
			
			Weil diese Aussage für \underline{alle} \(m,n \geq N_\varepsilon\) gilt, gilt sie auch für\\ \(n=N_\varepsilon,m\geq N_\varepsilon\). Das bedeutet, dass alle \(b_m\) nicht weiter von \(b_{N_\varepsilon}\) entfernt sind als \(\varepsilon\). Oder auch
			\begin{aleq*}
				\forall \varepsilon > 0 \colon \exists N_\varepsilon \in \mathbb{N} \colon \forall m \geq N_\varepsilon \colon |b_m - b_{N_\varepsilon}| < \varepsilon \text{.}
			\end{aleq*}
			
			Diese Aussage sei nicht zu verwechseln mit der Definition der Konvergenz. Der entscheidende Unterschied ist, dass \(a_{N_\varepsilon}\) kein fester Wert ist. Was allerdings aus dieser Aussage folgt ist, dass \((b_m)\) für \(m \geq N_\varepsilon\) beschränkt ist. Die Folge ist auch für alle \(m < N_\varepsilon\) beschränkt, weil es nur endlich viele Folgeglieder mit diesem Kriterium gibt. Es lässt sich also eine obere Schranke als \(max \left\lbrace b_n | n < N_\varepsilon \right\rbrace\) und eine untere Schranke als \(min \left\lbrace b_n | n < N_\varepsilon \right\rbrace\) berechnen.
			\TODO
		\end{proof}
	\end{thm}
	
	\TODO
	\subsection{Einschachtelungssatz/Sandwichlemma}
	\begin{thm}[Einschachtelungssatz/Sandwichlemma]
		Seien \((a_n)_{n=1}^\infty\), \((b_n)_{n=1}^\infty\) und \((c_n)_{n=1}^\infty\) beliebige Folgen mit \(\forall n \in \mathbb{N} \colon a_n \leq b_n \leq c_n\) und \(\lim\limits_{n \to \infty} a_n = \lim\limits_{n \to \infty} b_n = g\). Dann konvergiert auch \((b_n)\) gegen \(g\).
		
		\begin{proof}
			Zu zeigen ist
			\begin{aleq*}
				\forall \varepsilon > 0 \colon \exists N_\varepsilon \in \mathbb{N} \colon \forall n > N_\varepsilon \colon |b_n - g| < \varepsilon \text{.}
			\end{aleq*}
			\par
			Gegeben ist
			\begin{aleq*}
				\label{gegebenSandwich}
				\forall \varepsilon > 0 \colon \exists N_\varepsilon, M_\varepsilon \in \mathbb{N} \colon \forall n > N_\varepsilon, m > M_\varepsilon \colon |a_n-g| < \varepsilon \land |c_m - g| < \varepsilon \text{.}
			\end{aleq*}
			\par
			Es existiert also für alle \(\varepsilon > 0\) ein \(N_\varepsilon\) und ein \(M_\varepsilon\), so dass für \(K_\varepsilon = max \left\lbrace N_\varepsilon, M_\varepsilon \right\rbrace\) gilt
			\begin{aleq*}
				\forall k > K_\varepsilon \colon |a_k-g| < \varepsilon \land |c_k - g| < \varepsilon \text{.}
			\end{aleq*}
			\par
			Das und \(\forall n \in \mathbb{N} \colon a_n \leq b_n \leq c_n\) implizieren
			\begin{aleq*}
				&\forall k > K_\varepsilon \colon -\varepsilon < a_k-g \leq b_k-g \leq c_k-g < \varepsilon \\
				\implies &\forall k > K_\varepsilon \colon -\varepsilon < b_k-g < \varepsilon \\
				\iff &\forall k > K_\varepsilon \colon |b_k - g| < \varepsilon
			\end{aleq*}
			\par
			\((b_n)\) ist also ebenfalls konvergent gegen \(g\).
		\end{proof}
	\end{thm}
	
	\subsection{Teilfolgekriterium}
	\begin{thm}[Teilfolgekriterium]
		Eine Folge \((a_n)_{n=1}^\infty\) konvergiert genau dann gegen \(g\), wenn jede Teilfolge von \((a_n)\) ebenfalls gegen \(g\) konvergiert.
		\begin{proof}
			\((\impliedby)\) Wenn jede Teilfolge von \((a_n)\) gegen \(g\) konvergiert, konvergiert auch \((a_n)\) gegen \(g\), weil \((a_n)\) eine Teilfolge von \((a_n)\) ist. \\
			\par
			\((\implies)\) Indirekter Beweis. \\
				\textbf{Annahme: } Es gibt eine Teilfolge, die nicht gegen \(g\) konvergiert. Dann existiert  eine streng monoton steigende Folge von natürlichen Zahlen \((n_k)_{k=1}^\infty\), so dass die Folge \(b_k = a_{n_k}\) eine Teilfolge von \((a_n)\) ist, für die gilt
			\begin{aleq}
				\label{nichtKonvergenz}
				&\lnot \forall \varepsilon > 0 \colon \exists K_\varepsilon \in \mathbb{N} \colon \forall k > K_\varepsilon \colon |b_k -g| < \varepsilon \\
				\iff &\lnot \forall \varepsilon > 0 \colon \exists K_\varepsilon \in \mathbb{N} \colon \forall k > K_\varepsilon \colon |a_{n_k} -g| < \varepsilon
			\end{aleq}
			\par
			Weil \((a_n)\) gegen \(g\) konvergiert, gilt
			\begin{aleq*}
				&\forall \varepsilon > 0 \colon \exists N_\varepsilon \in \mathbb{N} \colon \forall m > N_\varepsilon \colon |a_m - g| < \varepsilon \text{.}
			\end{aleq*}
			\par
			Weil \(n_m \geq m > N_\varepsilon\) (strenge Monotonie von \((n_k)\)), gilt
			\begin{aleq*}
				&\forall \varepsilon > 0 \colon \exists N_\varepsilon \in \mathbb{N} \colon \forall m > N_\varepsilon \colon |a_{n_m} - g| < \varepsilon \text{,}
			\end{aleq*}
			\par
			was im Widerspruch zu (\ref{nichtKonvergenz}) steht.
 		\end{proof}
	\end{thm}
	
	\subsection{Grenzwertsätze für Folgen}
	\begin{thm}[Grenzwertsätze]
		\label{satzGW}
		Seien \((a_n)\) und \((b_n)\) konvergente Folgen mit
		\begin{aleq*}
			\lim_{n \to \infty} a_n = a \land \lim_{n \to \infty} b_n = b \text{.}
		\end{aleq*}
		
		Dann gelten folgende Aussagen.
		\begin{enumerate}
			\item \(\lim\limits_{n \to \infty} a_n + b_n = a + b\)
			\item \(\lim\limits_{n \to \infty} a_n - b_n = a - b\)
			\item \(\lim\limits_{n \to \infty} a_n * b_n = a * b\)
			\item \(\lim\limits_{n \to \infty} \frac{a_n}{b_n} = \frac{a}{b}, b \neq 0 \land \exists N_\varepsilon \colon \forall n \geq N_\varepsilon \colon b_n \neq 0\)
		\end{enumerate}
		
		\begin{proof}
			\quad\newline
			\begin{enumerate}
				\item Weil \((a_n)\) und \((b_n)\) gegen \(a\) bzw. \(b\) konvergieren (\(|a_n -a|\) und \(|b_n - b|\) werden beliebig klein), gilt
				\begin{aleq*}
					&\forall \varepsilon > 0 \colon \exists N_\varepsilon \colon \forall n \geq N_\varepsilon \colon |a_n - a | + |b_n - b| < \varepsilon \\
					\implies &\forall \varepsilon > 0 \colon \exists N_\varepsilon \colon \forall n \geq N_\varepsilon \colon |a_n - a + b_n - b| < \varepsilon && \text{Lemma \ref{dreiecksungleichung}} \\
					\iff &\forall \varepsilon > 0 \colon \exists N_\varepsilon \colon \forall n \geq N_\varepsilon \colon |a_n + b_n - (a + b)| < \varepsilon
				\end{aleq*}
				
				Das bedeutet, dass der Grenzwert von \(a_n + b_n = a+b\) ist. Damit ist 1. bewiesen.
				
				\item 
				Weil \((a_n)\) und \((b_n)\) gegen \(a\) bzw. \(b\) konvergieren, gilt analog zu 1.:
				\begin{aleq*}
					&\forall \varepsilon > 0 \colon \exists N_\varepsilon \colon \forall n \geq N_\varepsilon \colon |a_n - a| + |b_n - b| < \varepsilon \\
					\implies &\forall \varepsilon > 0 \colon \exists N_\varepsilon \colon \forall n \geq N_\varepsilon \colon |(a_n - a) - (b_n - b)| \leq |a_n - a| + |b_n - b| < \varepsilon && \text{Lemma \ref{dreiecksungleichung}} \\
					\iff &\forall \varepsilon > 0 \colon \exists N_\varepsilon \colon \forall n \geq N_\varepsilon \colon |(a_n - b_n) - (a - b)| < \varepsilon
				\end{aleq*}
				Damit ist der Grenzwert von \(a_n - b_n = a - b\). 2. ist bewiesen.
				
				\item \TODO
				
				\item \TODO
			\end{enumerate}
		\end{proof}
	\end{thm}
	
	\subsection{Übungsaufgaben}
	\begin{enumerate}
		\item Finden Sie den Grenzwert der jeweiligen Folge.
		\begin{enumerate}[label=(\alph*)]
			\item \(a_n = \frac{\frac{1}{n} + \frac{1}{n^2} + 1}{\frac{1}{n^3} + \frac{7}{n} + 3}\)
			\item \(b_n = \frac{n^2 + n - 1}{n^3 + 1}\)
			\item \(c_n = \frac{\left(\frac{1}{3}\right)^n + \left(\frac{1}{4}\right)^n + 1^n}{\left(\frac{1}{5}\right)^n + 2}\)
			\item \(d_n = \frac{3^n + 5^n + 7^n}{2^n + 3^n + 7^n}\)
		\end{enumerate}

		\item \needspace{2\baselineskip} Begründen Sie folgende Sätze mithilfe einer Skizze oder durch einen Beweis.
		\begin{enumerate}[label=(\alph*)]
			\item Eine Folge kann nur einen Grenzwert haben. \textit{(Hinweis: Indirekter Beweis)}
			\item Sei \((a_n)\) eine beliebige Folge für die \(\forall n \in \mathbb{N} \colon a_n > 0\) gilt (\((a_n)\) ist nach unten beschränkt, wobei eine untere Schranke \(0\) ist). Falls die Folge \((b_n)\) mit der Bildungsvorschrift \(b_n = (a_n)^n\) konvergiert kann \(\lim\limits_{n \to \infty} a_n < 0\) nicht stimmen.
		\end{enumerate}
	\end{enumerate}
	
	\subsubsection{Lösungen}
	\begin{enumerate}
		\item \quad
		\begin{enumerate}[label=(\alph*)]
			\item Wir wissen, dass \(\frac{1}{n}\) mit \(n \in \mathbb{N}\) gegen 0 konvergiert (Satz \ref{lim1/n}).
			\begin{aleq*}
				\lim_{n \to \infty} a_n = &\lim_{n \to \infty} \frac{\frac{1}{n} + \frac{1}{n^2} + 1}{\frac{1}{n^3} + \frac{7}{n} + 3} && \text{Satz \ref{lim1/n}} \\
				= &\lim_{n \to \infty} \frac{1}{3} && \text{\hyperref[satzGW]{Grenzwertsätze}} \\
				= &\frac{1}{3}
			\end{aleq*}
		\end{enumerate}
	\end{enumerate}
	\TODO
	
	\subsection{Bestimmte Divergenz}
	\begin{defn}[Bestimmte Dviergenz]
		Die Folge \((a_n)_{n=1}^\infty\) heißt genau dann \underline{bestimmt} divergent gegen \(\infty\), wenn gilt
		\begin{aleq*}
			\forall k \colon \exists N_k \colon \forall n \geq N_k \colon a_n > k \text{.}
		\end{aleq*}
		
		\((a_n)\) heißt bestimmt divergent gegen \(- \infty\) genau dann, wenn gilt
		\begin{aleq*}
			\forall k \colon \exists N_k \colon \forall n \geq N_k \colon a_n < k \text{.}
		\end{aleq*}
	\end{defn}
	
	\begin{thm}[Rechenregeln für bestimmte Divergenz (RRbD)]
		Seien \((a_n)\) und \((b_n)\) zwei beliebe Folgen, die bestimmt gegen \(\infty\) konvergieren und \((c_n)\) eine beliebe Folge, die gegen ein beliebiges \(c \in \mathbb{R}\) konvergiert. Dann gelten folgende Aussagen.
		\begin{enumerate}
			\item \(\lim\limits_{n \to \infty} a_n + b_n = \infty\)
			\item \(\lim\limits_{n \to \infty} -a_n - b_n = -\infty\)
			\item \(\lim\limits_{n \to \infty} a_n * b_n = \infty\)
			\item \(\lim\limits_{n \to \infty} a_n * (- b_n) = -\infty\)
			\item \(\lim\limits_{n \to \infty} a_n + c_n = \infty\)
			\item \(\lim\limits_{n \to \infty} a_n * c_n = 
			\begin{cases}
				\begin{aligned}
					\infty, &&c > 0 \\
					-\infty, &&c < 0
				\end{aligned}
			\end{cases}\)
			\item \(\lim\limits_{n \to \infty} \frac{a_n}{c_n} = 
			\begin{cases}
				\begin{aligned}
					\infty, &&&c > 0 \\
					-\infty, &&&c< 0 \\
					\infty, &&&c = 0 \land \forall n \in \mathbb{N} \colon a_n > 0 \\
					-\infty, &&&c = 0 \land \forall n \in \mathbb{N} \colon a_n < 0
				\end{aligned}
			\end{cases}\)
			\item \(\lim\limits_{n \to \infty} \frac{c_n}{a_n} = 0\)
		\end{enumerate}
	\end{thm}
	
	\section{Grenzwerte und Funktionen}
	\subsection{Stetigkeit}
	\begin{defn}[Stetigkeit über Folgen]
		Eine Funktion \(f \colon A \to \mathbb{R}\) heißt stetig in \(x_0 \in A\) genau dann, wenn für alle Folgen \((x_n)_{n=1}^\infty\) mit 
		\begin{aleq*}
			&\forall n \colon x_n \in A \\
			\land &\lim_{n \to \infty} x_n = x_0
		\end{aleq*}
		
		gilt \(\lim\limits_{n \to \infty}  f(x_n) = f(x_0)\).
	\end{defn}
	
	\begin{defn}[\(\varepsilon-\delta\)-Definition für Stetigkeit]
		Eine Funktion \(f \colon A \to \mathbb{R}\) heißt stetig in \(x_0 \in A\) genau dann, wenn
		\begin{aleq*}
			\forall \varepsilon > 0 \colon \exists \delta > 0 \colon \forall x \in A \setminus \left\lbrace x_0 \right\rbrace \colon |x_0 - x| < \delta \implies |f(x_0) - f(x)| < \varepsilon \text{.}
		\end{aleq*}
	\end{defn}
	
	\newpage
	\part{Q2}
	\part{Q3}
	\part{Q4}
	
	\part{Anhang}
	\section{\rlap{GNU Free Documentation License}}
	\label{sec:fdl}
	%\label{label_fdl}
	
	\begin{center}
		
		Version 1.3, 3 November 2008
		
		
		Copyright \copyright{} 2000, 2001, 2002, 2007, 2008  Free Software Foundation, Inc.
		
		\bigskip
		
		\href{https://fsf.org/}{\texttt{<https://fsf.org/>}}
		
		\bigskip
		
		Everyone is permitted to copy and distribute verbatim copies
		of this license document, but changing it is not allowed.
	\end{center}
	
	
	\begin{center}
		{\bf\large Preamble}
	\end{center}
	
	The purpose of this License is to make a manual, textbook, or other
	functional and useful document ``free'' in the sense of freedom: to
	assure everyone the effective freedom to copy and redistribute it,
	with or without modifying it, either commercially or noncommercially.
	Secondarily, this License preserves for the author and publisher a way
	to get credit for their work, while not being considered responsible
	for modifications made by others.
	
	This License is a kind of ``copyleft'', which means that derivative
	works of the document must themselves be free in the same sense.  It
	complements the GNU General Public License, which is a copyleft
	license designed for free software.
	
	We have designed this License in order to use it for manuals for free
	software, because free software needs free documentation: a free
	program should come with manuals providing the same freedoms that the
	software does.  But this License is not limited to software manuals;
	it can be used for any textual work, regardless of subject matter or
	whether it is published as a printed book.  We recommend this License
	principally for works whose purpose is instruction or reference.
	
	
	\begin{center}
		{\Large\bf 1. APPLICABILITY AND DEFINITIONS\par}
		\phantomsection
		\addcontentsline{toc}{subsection}{1. APPLICABILITY AND DEFINITIONS}
	\end{center}
	
	This License applies to any manual or other work, in any medium, that
	contains a notice placed by the copyright holder saying it can be
	distributed under the terms of this License.  Such a notice grants a
	world-wide, royalty-free license, unlimited in duration, to use that
	work under the conditions stated herein.  The ``\textbf{Document}'', below,
	refers to any such manual or work.  Any member of the public is a
	licensee, and is addressed as ``\textbf{you}''.  You accept the license if you
	copy, modify or distribute the work in a way requiring permission
	under copyright law.
	
	A ``\textbf{Modified Version}'' of the Document means any work containing the
	Document or a portion of it, either copied verbatim, or with
	modifications and/or translated into another language.
	
	A ``\textbf{Secondary Section}'' is a named appendix or a front-matter section of
	the Document that deals exclusively with the relationship of the
	publishers or authors of the Document to the Document's overall subject
	(or to related matters) and contains nothing that could fall directly
	within that overall subject.  (Thus, if the Document is in part a
	textbook of mathematics, a Secondary Section may not explain any
	mathematics.)  The relationship could be a matter of historical
	connection with the subject or with related matters, or of legal,
	commercial, philosophical, ethical or political position regarding
	them.
	
	The ``\textbf{Invariant Sections}'' are certain Secondary Sections whose titles
	are designated, as being those of Invariant Sections, in the notice
	that says that the Document is released under this License.  If a
	section does not fit the above definition of Secondary then it is not
	allowed to be designated as Invariant.  The Document may contain zero
	Invariant Sections.  If the Document does not identify any Invariant
	Sections then there are none.
	
	The ``\textbf{Cover Texts}'' are certain short passages of text that are listed,
	as Front-Cover Texts or Back-Cover Texts, in the notice that says that
	the Document is released under this License.  A Front-Cover Text may
	be at most 5 words, and a Back-Cover Text may be at most 25 words.
	
	A ``\textbf{Transparent}'' copy of the Document means a machine-readable copy,
	represented in a format whose specification is available to the
	general public, that is suitable for revising the document
	straightforwardly with generic text editors or (for images composed of
	pixels) generic paint programs or (for drawings) some widely available
	drawing editor, and that is suitable for input to text formatters or
	for automatic translation to a variety of formats suitable for input
	to text formatters.  A copy made in an otherwise Transparent file
	format whose markup, or absence of markup, has been arranged to thwart
	or discourage subsequent modification by readers is not Transparent.
	An image format is not Transparent if used for any substantial amount
	of text.  A copy that is not ``Transparent'' is called ``\textbf{Opaque}''.
	
	Examples of suitable formats for Transparent copies include plain
	ASCII without markup, Texinfo input format, LaTeX input format, SGML
	or XML using a publicly available DTD, and standard-conforming simple
	HTML, PostScript or PDF designed for human modification.  Examples of
	transparent image formats include PNG, XCF and JPG.  Opaque formats
	include proprietary formats that can be read and edited only by
	proprietary word processors, SGML or XML for which the DTD and/or
	processing tools are not generally available, and the
	machine-generated HTML, PostScript or PDF produced by some word
	processors for output purposes only.
	
	The ``\textbf{Title Page}'' means, for a printed book, the title page itself,
	plus such following pages as are needed to hold, legibly, the material
	this License requires to appear in the title page.  For works in
	formats which do not have any title page as such, ``Title Page'' means
	the text near the most prominent appearance of the work's title,
	preceding the beginning of the body of the text.
	
	The ``\textbf{publisher}'' means any person or entity that distributes
	copies of the Document to the public.
	
	A section ``\textbf{Entitled XYZ}'' means a named subunit of the Document whose
	title either is precisely XYZ or contains XYZ in parentheses following
	text that translates XYZ in another language.  (Here XYZ stands for a
	specific section name mentioned below, such as ``\textbf{Acknowledgements}'',
	``\textbf{Dedications}'', ``\textbf{Endorsements}'', or ``\textbf{History}''.)  
	To ``\textbf{Preserve the Title}''
	of such a section when you modify the Document means that it remains a
	section ``Entitled XYZ'' according to this definition.
	
	The Document may include Warranty Disclaimers next to the notice which
	states that this License applies to the Document.  These Warranty
	Disclaimers are considered to be included by reference in this
	License, but only as regards disclaiming warranties: any other
	implication that these Warranty Disclaimers may have is void and has
	no effect on the meaning of this License.
	
	
	\begin{center}
		{\Large\bf 2. VERBATIM COPYING\par}
		\phantomsection
		\addcontentsline{toc}{subsection}{2. VERBATIM COPYING}
	\end{center}
	
	You may copy and distribute the Document in any medium, either
	commercially or noncommercially, provided that this License, the
	copyright notices, and the license notice saying this License applies
	to the Document are reproduced in all copies, and that you add no other
	conditions whatsoever to those of this License.  You may not use
	technical measures to obstruct or control the reading or further
	copying of the copies you make or distribute.  However, you may accept
	compensation in exchange for copies.  If you distribute a large enough
	number of copies you must also follow the conditions in section~3.
	
	You may also lend copies, under the same conditions stated above, and
	you may publicly display copies.
	
	
	\begin{center}
		{\Large\bf 3. COPYING IN QUANTITY\par}
		\phantomsection
		\addcontentsline{toc}{subsection}{3. COPYING IN QUANTITY}
	\end{center}
	
	
	If you publish printed copies (or copies in media that commonly have
	printed covers) of the Document, numbering more than 100, and the
	Document's license notice requires Cover Texts, you must enclose the
	copies in covers that carry, clearly and legibly, all these Cover
	Texts: Front-Cover Texts on the front cover, and Back-Cover Texts on
	the back cover.  Both covers must also clearly and legibly identify
	you as the publisher of these copies.  The front cover must present
	the full title with all words of the title equally prominent and
	visible.  You may add other material on the covers in addition.
	Copying with changes limited to the covers, as long as they preserve
	the title of the Document and satisfy these conditions, can be treated
	as verbatim copying in other respects.
	
	If the required texts for either cover are too voluminous to fit
	legibly, you should put the first ones listed (as many as fit
	reasonably) on the actual cover, and continue the rest onto adjacent
	pages.
	
	If you publish or distribute Opaque copies of the Document numbering
	more than 100, you must either include a machine-readable Transparent
	copy along with each Opaque copy, or state in or with each Opaque copy
	a computer-network location from which the general network-using
	public has access to download using public-standard network protocols
	a complete Transparent copy of the Document, free of added material.
	If you use the latter option, you must take reasonably prudent steps,
	when you begin distribution of Opaque copies in quantity, to ensure
	that this Transparent copy will remain thus accessible at the stated
	location until at least one year after the last time you distribute an
	Opaque copy (directly or through your agents or retailers) of that
	edition to the public.
	
	It is requested, but not required, that you contact the authors of the
	Document well before redistributing any large number of copies, to give
	them a chance to provide you with an updated version of the Document.
	
	
	\begin{center}
		{\Large\bf 4. MODIFICATIONS\par}
		\phantomsection
		\addcontentsline{toc}{subsection}{4. MODIFICATIONS}
	\end{center}
	
	You may copy and distribute a Modified Version of the Document under
	the conditions of sections 2 and 3 above, provided that you release
	the Modified Version under precisely this License, with the Modified
	Version filling the role of the Document, thus licensing distribution
	and modification of the Modified Version to whoever possesses a copy
	of it.  In addition, you must do these things in the Modified Version:
	
	\begin{itemize}
		\item[A.] 
		Use in the Title Page (and on the covers, if any) a title distinct
		from that of the Document, and from those of previous versions
		(which should, if there were any, be listed in the History section
		of the Document).  You may use the same title as a previous version
		if the original publisher of that version gives permission.
		
		\item[B.]
		List on the Title Page, as authors, one or more persons or entities
		responsible for authorship of the modifications in the Modified
		Version, together with at least five of the principal authors of the
		Document (all of its principal authors, if it has fewer than five),
		unless they release you from this requirement.
		
		\item[C.]
		State on the Title page the name of the publisher of the
		Modified Version, as the publisher.
		
		\item[D.]
		Preserve all the copyright notices of the Document.
		
		\item[E.]
		Add an appropriate copyright notice for your modifications
		adjacent to the other copyright notices.
		
		\item[F.]
		Include, immediately after the copyright notices, a license notice
		giving the public permission to use the Modified Version under the
		terms of this License, in the form shown in the Addendum below.
		
		\item[G.]
		Preserve in that license notice the full lists of Invariant Sections
		and required Cover Texts given in the Document's license notice.
		
		\item[H.]
		Include an unaltered copy of this License.
		
		\item[I.]
		Preserve the section Entitled ``History'', Preserve its Title, and add
		to it an item stating at least the title, year, new authors, and
		publisher of the Modified Version as given on the Title Page.  If
		there is no section Entitled ``History'' in the Document, create one
		stating the title, year, authors, and publisher of the Document as
		given on its Title Page, then add an item describing the Modified
		Version as stated in the previous sentence.
		
		\item[J.]
		Preserve the network location, if any, given in the Document for
		public access to a Transparent copy of the Document, and likewise
		the network locations given in the Document for previous versions
		it was based on.  These may be placed in the ``History'' section.
		You may omit a network location for a work that was published at
		least four years before the Document itself, or if the original
		publisher of the version it refers to gives permission.
		
		\item[K.]
		For any section Entitled ``Acknowledgements'' or ``Dedications'',
		Preserve the Title of the section, and preserve in the section all
		the substance and tone of each of the contributor acknowledgements
		and/or dedications given therein.
		
		\item[L.]
		Preserve all the Invariant Sections of the Document,
		unaltered in their text and in their titles.  Section numbers
		or the equivalent are not considered part of the section titles.
		
		\item[M.]
		Delete any section Entitled ``Endorsements''.  Such a section
		may not be included in the Modified Version.
		
		\item[N.]
		Do not retitle any existing section to be Entitled ``Endorsements''
		or to conflict in title with any Invariant Section.
		
		\item[O.]
		Preserve any Warranty Disclaimers.
	\end{itemize}
	
	If the Modified Version includes new front-matter sections or
	appendices that qualify as Secondary Sections and contain no material
	copied from the Document, you may at your option designate some or all
	of these sections as invariant.  To do this, add their titles to the
	list of Invariant Sections in the Modified Version's license notice.
	These titles must be distinct from any other section titles.
	
	You may add a section Entitled ``Endorsements'', provided it contains
	nothing but endorsements of your Modified Version by various
	parties---for example, statements of peer review or that the text has
	been approved by an organization as the authoritative definition of a
	standard.
	
	You may add a passage of up to five words as a Front-Cover Text, and a
	passage of up to 25 words as a Back-Cover Text, to the end of the list
	of Cover Texts in the Modified Version.  Only one passage of
	Front-Cover Text and one of Back-Cover Text may be added by (or
	through arrangements made by) any one entity.  If the Document already
	includes a cover text for the same cover, previously added by you or
	by arrangement made by the same entity you are acting on behalf of,
	you may not add another; but you may replace the old one, on explicit
	permission from the previous publisher that added the old one.
	
	The author(s) and publisher(s) of the Document do not by this License
	give permission to use their names for publicity for or to assert or
	imply endorsement of any Modified Version.
	
	
	\begin{center}
		{\Large\bf 5. COMBINING DOCUMENTS\par}
		\phantomsection
		\addcontentsline{toc}{subsection}{5. COMBINING DOCUMENTS}
	\end{center}
	
	
	You may combine the Document with other documents released under this
	License, under the terms defined in section~4 above for modified
	versions, provided that you include in the combination all of the
	Invariant Sections of all of the original documents, unmodified, and
	list them all as Invariant Sections of your combined work in its
	license notice, and that you preserve all their Warranty Disclaimers.
	
	The combined work need only contain one copy of this License, and
	multiple identical Invariant Sections may be replaced with a single
	copy.  If there are multiple Invariant Sections with the same name but
	different contents, make the title of each such section unique by
	adding at the end of it, in parentheses, the name of the original
	author or publisher of that section if known, or else a unique number.
	Make the same adjustment to the section titles in the list of
	Invariant Sections in the license notice of the combined work.
	
	In the combination, you must combine any sections Entitled ``History''
	in the various original documents, forming one section Entitled
	``History''; likewise combine any sections Entitled ``Acknowledgements'',
	and any sections Entitled ``Dedications''.  You must delete all sections
	Entitled ``Endorsements''.
	
	\begin{center}
		{\Large\bf 6. COLLECTIONS OF DOCUMENTS\par}
		\phantomsection
		\addcontentsline{toc}{subsection}{6. COLLECTIONS OF DOCUMENTS}
	\end{center}
	
	You may make a collection consisting of the Document and other documents
	released under this License, and replace the individual copies of this
	License in the various documents with a single copy that is included in
	the collection, provided that you follow the rules of this License for
	verbatim copying of each of the documents in all other respects.
	
	You may extract a single document from such a collection, and distribute
	it individually under this License, provided you insert a copy of this
	License into the extracted document, and follow this License in all
	other respects regarding verbatim copying of that document.
	
	
	\begin{center}
		{\Large\bf 7. AGGREGATION WITH INDEPENDENT WORKS\par}
		\phantomsection
		\addcontentsline{toc}{subsection}{7. AGGREGATION WITH INDEPENDENT WORKS}
	\end{center}
	
	
	A compilation of the Document or its derivatives with other separate
	and independent documents or works, in or on a volume of a storage or
	distribution medium, is called an ``aggregate'' if the copyright
	resulting from the compilation is not used to limit the legal rights
	of the compilation's users beyond what the individual works permit.
	When the Document is included in an aggregate, this License does not
	apply to the other works in the aggregate which are not themselves
	derivative works of the Document.
	
	If the Cover Text requirement of section~3 is applicable to these
	copies of the Document, then if the Document is less than one half of
	the entire aggregate, the Document's Cover Texts may be placed on
	covers that bracket the Document within the aggregate, or the
	electronic equivalent of covers if the Document is in electronic form.
	Otherwise they must appear on printed covers that bracket the whole
	aggregate.
	
	
	\begin{center}
		{\Large\bf 8. TRANSLATION\par}
		\phantomsection
		\addcontentsline{toc}{subsection}{8. TRANSLATION}
	\end{center}
	
	
	Translation is considered a kind of modification, so you may
	distribute translations of the Document under the terms of section~4.
	Replacing Invariant Sections with translations requires special
	permission from their copyright holders, but you may include
	translations of some or all Invariant Sections in addition to the
	original versions of these Invariant Sections.  You may include a
	translation of this License, and all the license notices in the
	Document, and any Warranty Disclaimers, provided that you also include
	the original English version of this License and the original versions
	of those notices and disclaimers.  In case of a disagreement between
	the translation and the original version of this License or a notice
	or disclaimer, the original version will prevail.
	
	If a section in the Document is Entitled ``Acknowledgements'',
	``Dedications'', or ``History'', the requirement (section~4) to Preserve
	its Title (section~1) will typically require changing the actual
	title.
	
	
	\begin{center}
		{\Large\bf 9. TERMINATION\par}
		\phantomsection
		\addcontentsline{toc}{subsection}{9. TERMINATION}
	\end{center}
	
	
	You may not copy, modify, sublicense, or distribute the Document
	except as expressly provided under this License.  Any attempt
	otherwise to copy, modify, sublicense, or distribute it is void, and
	will automatically terminate your rights under this License.
	
	However, if you cease all violation of this License, then your license
	from a particular copyright holder is reinstated (a) provisionally,
	unless and until the copyright holder explicitly and finally
	terminates your license, and (b) permanently, if the copyright holder
	fails to notify you of the violation by some reasonable means prior to
	60 days after the cessation.
	
	Moreover, your license from a particular copyright holder is
	reinstated permanently if the copyright holder notifies you of the
	violation by some reasonable means, this is the first time you have
	received notice of violation of this License (for any work) from that
	copyright holder, and you cure the violation prior to 30 days after
	your receipt of the notice.
	
	Termination of your rights under this section does not terminate the
	licenses of parties who have received copies or rights from you under
	this License.  If your rights have been terminated and not permanently
	reinstated, receipt of a copy of some or all of the same material does
	not give you any rights to use it.
	
	
	\begin{center}
		{\Large\bf 10. FUTURE REVISIONS OF THIS LICENSE\par}
		\phantomsection
		\addcontentsline{toc}{subsection}{10. FUTURE REVISIONS OF THIS LICENSE}
	\end{center}
	
	
	The Free Software Foundation may publish new, revised versions
	of the GNU Free Documentation License from time to time.  Such new
	versions will be similar in spirit to the present version, but may
	differ in detail to address new problems or concerns.  See
	\href{https://www.gnu.org/licenses/}{\texttt{https://www.gnu.org/licenses/}}.
	
	Each version of the License is given a distinguishing version number.
	If the Document specifies that a particular numbered version of this
	License ``or any later version'' applies to it, you have the option of
	following the terms and conditions either of that specified version or
	of any later version that has been published (not as a draft) by the
	Free Software Foundation.  If the Document does not specify a version
	number of this License, you may choose any version ever published (not
	as a draft) by the Free Software Foundation.  If the Document
	specifies that a proxy can decide which future versions of this
	License can be used, that proxy's public statement of acceptance of a
	version permanently authorizes you to choose that version for the
	Document.
	
	
	\begin{center}
		{\Large\bf 11. RELICENSING\par}
		\phantomsection
		\addcontentsline{toc}{subsection}{11. RELICENSING}
	\end{center}
	
	
	``Massive Multiauthor Collaboration Site'' (or ``MMC Site'') means any
	World Wide Web server that publishes copyrightable works and also
	provides prominent facilities for anybody to edit those works.  A
	public wiki that anybody can edit is an example of such a server.  A
	``Massive Multiauthor Collaboration'' (or ``MMC'') contained in the
	site means any set of copyrightable works thus published on the MMC
	site.
	
	``CC-BY-SA'' means the Creative Commons Attribution-Share Alike 3.0
	license published by Creative Commons Corporation, a not-for-profit
	corporation with a principal place of business in San Francisco,
	California, as well as future copyleft versions of that license
	published by that same organization.
	
	``Incorporate'' means to publish or republish a Document, in whole or
	in part, as part of another Document.
	
	An MMC is ``eligible for relicensing'' if it is licensed under this
	License, and if all works that were first published under this License
	somewhere other than this MMC, and subsequently incorporated in whole
	or in part into the MMC, (1) had no cover texts or invariant sections,
	and (2) were thus incorporated prior to November 1, 2008.
	
	The operator of an MMC Site may republish an MMC contained in the site
	under CC-BY-SA on the same site at any time before August 1, 2009,
	provided the MMC is eligible for relicensing.
	
	
	\begin{center}
		{\Large\bf ADDENDUM: How to use this License for your documents\par}
		\phantomsection
		\addcontentsline{toc}{subsection}{ADDENDUM: How to use this License for your documents}
	\end{center}
	
	To use this License in a document you have written, include a copy of
	the License in the document and put the following copyright and
	license notices just after the title page:
	
	\bigskip
	\begin{quote}
		Copyright \copyright{}  YEAR  YOUR NAME.
		Permission is granted to copy, distribute and/or modify this document
		under the terms of the GNU Free Documentation License, Version 1.3
		or any later version published by the Free Software Foundation;
		with no Invariant Sections, no Front-Cover Texts, and no Back-Cover Texts.
		A copy of the license is included in the section entitled ``GNU
		Free Documentation License''.
	\end{quote}
	\bigskip
	
	If you have Invariant Sections, Front-Cover Texts and Back-Cover Texts,
	replace the ``with \dots\ Texts.''\ line with this:
	
	\bigskip
	\begin{quote}
		with the Invariant Sections being LIST THEIR TITLES, with the
		Front-Cover Texts being LIST, and with the Back-Cover Texts being LIST.
	\end{quote}
	\bigskip
	
	If you have Invariant Sections without Cover Texts, or some other
	combination of the three, merge those two alternatives to suit the
	situation.
	
	If your document contains nontrivial examples of program code, we
	recommend releasing these examples in parallel under your choice of
	free software license, such as the GNU General Public License,
	to permit their use in free software.
\end{document}