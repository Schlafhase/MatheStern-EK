\documentclass{article}
\usepackage{amsmath, amsfonts, amssymb, amsthm, array, enumitem}
\usepackage[parfill]{parskip}
\usepackage[german]{babel}

\newcommand\func[5]{%
	\begingroup
	\setlength\arraycolsep{0pt}
	#1\colon\begin{array}[t]{l >{{}}c<{{}} l}
		#2 & \to & #3 \\ #4 & \mapsto & #5 
	\end{array}%
	\endgroup}

\newtheorem{thm}{Satz}[section]
\newtheorem{lem}[thm]{Lemma}
\newtheorem{cor}[thm]{Schlussfolgerung}
\newtheorem{rem}[thm]{Bemerkung}
\newtheorem{remark}[thm]{Bemerkung}
\newtheorem{conj}[thm]{Annahme}
\newtheorem{defn}{Definition}[section]
\renewcommand\qedsymbol{QED}

\makeatletter
\newcommand*{\rom}[1]{\expandafter\@slowromancap\romannumeral #1@}
\makeatother

\newenvironment{aleq}{\begin{equation}\begin{aligned}}{\end{aligned}\end{equation}}
\newenvironment{aleq*}{\begin{equation*}\begin{aligned}}{\end{aligned}\end{equation*}}

\title{Mathematik*}
\author{Linus Yury Schneeberg}
\date{2025/26}
\begin{document}
	\maketitle
	\tableofcontents
	
	\vspace*{\fill}
	\textbf{Hinweis: }Dieses Dokument ist unvollständig und enthält nur einige Aufzeichnungen aus dem Mathe*-Kurs bei Herrn Ohnesorge. Drei vertikal übereinanderstehende Punkte stehen dafür, dass Abschnitte ausgelassen wurden.
	\newpage
	
	
	\part{Q1}
	\section{Differentialrechnung}
	\vdots
	\subsection{Ableitungen in verschiedenen Kontexten}
	\textbf{reine Mathematik} \\
	Bestimmen Sie die Ableitung der folgenden Fuktionen an einer beliebigen Stelle \(x_0\) ihres Definitionsbereiches:
	
	\begin{enumerate}[label=(\alph*)]
		\item \(c(x) = c, x \in \mathbb{R} \land c \in \mathbb{R}\)
		\\
		Dann ist
		\[c'(x_0) = \lim_{h \to 0} \frac{c(x_0+h) - c(x_0)}{h}\text{.}\]
		Das lässt sich zu
		\(c'(x_0) = \lim\limits_{h \to 0} 0\)
		umformen.
		\begin{aleq*}
			&c'(x_0) = \lim_{h \to 0} 0\\
			\text{genau dann, wenn: } &\forall \varepsilon > 0 \colon \exists \delta>0 \colon\forall x \colon |0-x| < \delta \implies |0 - c'(x_0)| < \varepsilon \\
			\text{genau dann, wenn: } &\forall \varepsilon > 0 \colon \exists \delta>0 \colon\forall x \colon |x| < \delta \implies |c'(x_0)| < \varepsilon \\
			\text{genau dann, wenn: } &\forall \varepsilon > 0 \colon \left[\left(\forall \delta > 0 \colon \exists x \colon |x|<\delta\right)\implies |c'(x_0)|<\varepsilon\right] && \text{Lemma 1.1, 1.2} \\
			\text{genau dann, wenn: } &\forall \varepsilon > 0 \colon |c'(x_0)| < \varepsilon
		\end{aleq*}
		\par
		Da \(\forall \varepsilon>0 \colon |c'(x_0)| < \varepsilon\) nur für \(c'(x_0) = 0\) gilt, ist \(c'(x_0) = 0\).
		\begin{lem}
			Sei \(P(x)\) eine logische Aussage, die abhängig von \(x\) ist und \(Q\) eine logische Aussage, die unabhängig von \(x\) ist. Dann gilt 
			\begin{aleq*}
				&\forall x \colon P(x) \implies Q \\
				\text{genau dann, wenn: } &\left(\exists x \colon P(x)\right) \implies Q
			\end{aleq*}
			\begin{proof}
				\begin{aleq*}
					&\forall x \colon P(x) \implies Q \\
					\text{genau dann, wenn: } &\forall x \colon \lnot P(x) \lor Q \\
					\text{genau dann, wenn: } &\bigwedge_{x \in \mathbb{R}} \lnot P(x) \lor Q \\
					\text{genau dann, wenn: } &\left(\bigwedge_{x \in \mathbb{R}} \lnot P(x)\right) \lor Q \\
					\text{genau dann, wenn: } &\left(\lnot \bigvee_{x \in \mathbb{R}} P(x)\right) \lor Q \\
					\text{genau dann, wenn: } &\lnot \left(\exists x \colon P(x)\right) \lor Q \\
					\text{genau dann, wenn: } &\left(\exists x \colon P(x) \right) \implies Q
				\end{aleq*}
			\end{proof}
		\end{lem}
		\begin{lem}
			Sei \(P(x)\) eine logische Aussage, die abhängig von \(x\) ist und \(Q\) eine logische Aussage, die unabhängig von \(x\) ist. Dann gilt 
			\begin{aleq*}
				&\exists x \colon P(x) \implies Q \\
				\text{genau dann, wenn: } &\left(\forall x \colon P(x)\right) \implies Q
			\end{aleq*}
			\begin{proof}
				\begin{aleq*}
					&\exists x \colon P(x) \implies Q \\
					\text{genau dann, wenn: } &\exists x \colon \lnot P(x) \lor Q \\
					\text{genau dann, wenn: } &\bigvee_{x \in \mathbb{R}} \lnot P(x) \lor Q \\
					\text{genau dann, wenn: } &\left(\bigvee_{x \in \mathbb{R}} \lnot P(x)\right) \lor Q \\
					\text{genau dann, wenn: } &\left(\lnot\bigwedge_{x \in \mathbb{R}} P(x)\right) \lor Q \\
					\text{genau dann, wenn: } &\left(\lnot\forall x \colon P(x)\right) \lor Q \\
					\text{genau dann, wenn: } &(\forall x \colon P(x)) \implies Q
				\end{aleq*}
			\end{proof}
		\end{lem}
		\item \(f(x) = ax + b, x,a,b \in \mathbb{R}\) \\
		Dann ist
		\begin{aleq*}
			f'(x_0) &= \lim_{h \to 0} \frac{f(x_0 + h) - f(x0)}{h} \\
			&= \lim_{h \to 0} \frac{a*(x_0+h)+b - (ax_0+b)}{h} \\
			&= \lim_{h \to 0} \frac{ah}{h} \\
			&= \lim_{h \to 0} a \\
			&= a
		\end{aleq*}
		\item \(p(x) = x^2, x \in \mathbb{R}\) \\
		Dann ist
		\begin{aleq*}
			p'(x_0) &= \lim_{h \to 0}\frac{p(x_0 + h) - p(x_0)}{h} \\
			&= \lim_{h \to 0} \frac{(x_0 + h)^2 - x_0^2}{h} \\
			&= \lim_{h \to 0} \frac{x_0^2 + 2x_0h + h^2 - x_0^2}{h} \\
			&= \lim_{h \to 0} 2x_0 + h \\
			&= 2x_0
		\end{aleq*}
		\item \(q(x) = x^3, x \in \mathbb{R}\) \\
		Dann ist \begin{aleq*}
			q'(x_0) &= \lim_{h \to 0} \frac{q(x_0 + h) - q(x_0)}{h} \\
			&= \lim_{h \to 0} \frac{(x_0 + h)^3 - x_0^3}{h} \\
			&= \lim_{h \to 0} \frac{3x_0h^2 + 3x_0^2h + h^3}{h} \\
			&= \lim_{h \to 0} 3x_0h + 3x_0^2 + h^2 \\
			&= 3x_0^3
		\end{aleq*}
		\item \(g(x) = x^n, x \in \mathbb{R} \land n \in \mathbb{N} \setminus \left\lbrace 0 \right\rbrace\) \\
		Dann ist \begin{aleq*}
			q'(x_0) &= \lim_{h \to 0} \frac{g(x_0 + h) - g(x_0)}{h} \\
			&= \lim_{h \to 0} \frac{(x_0 + h)^n - x_0^n}{h} \\
			&\stackrel{?}{=} nx_0^2
		\end{aleq*}
	\end{enumerate}
	
\end{document}
